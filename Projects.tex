    %--------------------------------------------------------Projects
    \begin{tcolorbox}[title=Projects \faLightbulb,colframe=blue!25!black]


        \textbf{Finance Project}: Diagnosis and Strategy to Enhance Post-Pandemic Results at FIFCO\\
        \faMapMarker \hspace{0mm} \href{https://www.fifco.com/}{FIFCO} \hspace{13.7cm} \faCalendar \hspace{0mm} 2020\\
        \textit{This project analyzes the financial impact of COVID-19 on FIFCO from 2019 to 2023. Using financial indicators such as EBITDA, EVA, and liquidity ratios, the study evaluates the company’s response to the crisis and its recovery strategy. Despite a sharp decline in 2020, FIFCO maintained financial stability through cost control, innovation, and international expansion. By 2022, it achieved record results. As an analyst, I learned the value of resilience, data-driven decisions, and strategic adaptability. The study concludes with recommendations to optimize inventory, protect cash flow, and strengthen stakeholder relationships.}
        
        
        \tcbline




        \textbf{Electrical Project}: Modeling and simulation of faults on electric power systems using OPAL-RT software\\
        \faMapMarker \hspace{0mm} \href{https://www.ucr.ac.cr/}{Universidad de Costa Rica} \hspace{10.70cm} \faCalendar \hspace{0mm} 2020\\
        \textit{In this project, I used two IEEE papers to build power system models with 4 and 37 buses. The main objective was to apply different types of faults in order to test the proper performance of the protection system. I built both models using the Hypersim software, developed by OPAL-RT. My faculty advisor for this project was \href{https://eie.ucr.ac.cr/profesores/oscar.nunez/}{Ing. Oscar Núñez Mata, PhD.}}    
        
        
        \tcbline
        
        \textbf{BeeTheChange}: Hardwarethon \\
        \faMapMarker \hspace{0mm} \href{https://www.depts.ttu.edu/costarica/}{Texas Tech University, Escazú, San José} \hspace{8.4cm} \faCalendar \hspace{0mm} 2018 (August)\\
        \textit{This project consisted of the design and implementation of a smart beehive for stingless bees. We visited a beekeeper in Puntarenas province to learn about the needs of the bees and the beekeeper. Then, we designed and built a prototype with the following functions:}
        \begin{itemize}
            \item \textit{Load sensor (for measuring honey production).}
            \item \textit{Sound sensor (for measuring how “active” the beehive was).}
            \item \textit{Thermographic camera (to view the interior of the beehive).}
            \item \textit{Stroke switch (to detect when the beehive was opened).}
        \end{itemize}
         
        \textit{All these sensors were connected to a Raspberry Pi. Finally, we prepared and delivered a presentation of the final product to the Hardwarethon judges.}

    \end{tcolorbox}
    